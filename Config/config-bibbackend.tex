%%%%%%%%%%%%%%%%%%%%%%%%%%%%%%%%%%%%%%%%%%%%%%%%%%%
% Bibliography backend control. It is recommended  that we use biblatex, as it
% supports more keys (for example, when we cite a website we can specify the
% visited date, in the .bib file). It also support multiple files more easily
% and more bibliography styles

% \newcommand{\bibliosystem}{bibtex} % Valid options are biblatex or bibtex
\newcommand{\bibliosystem}{biblatex} % Valid options are biblatex or bibtex

\ifthenelse{\equal{\bibliosystem}{biblatex}}
{
  % Suggestion by Miguel Cubero (2023)
  % When using babel or polyglossia with biblatex, loading csquotes is
  % recommended to ensure that quoted texts are typeset according to the
  % rules of your main language.

  \usepackage{csquotes}

  % Use biblatex instead of bibtex
  \usepackage[backend=biber,style=ieee]{biblatex}
  % This is a dirty hack, but should work... The reason to do so is to avoid
  % the need of editing this file by the user (see Book/biblio files for more
  % details)
  \input{../Book/biblio/bibliofiles.tex}


  \ifdef{\mybibfileOne}
  {
    \addbibresource{\myreferencespath\mybibfileOne}
  }
  {
    \errorYOUmustDEFINEatLEASTmybibfileOneInbibliofilesDOTtex
  }
  \ifdef{\mybibfileTwo}
  {
    \addbibresource{\myreferencespath\mybibfileTwo}
  }
  {
  }
  \ifdef{\mybibfileThree}
  {
    \addbibresource{\myreferencespath\mybibfileThree}
  }
  {
  }
  \ifdef{\mybibfileFour}
  {
    \addbibresource{\myreferencespath\mybibfileFour}
  }
  {
  }
  \ifdef{\mybibfileFive}
  {
  \addbibresource{\myreferencespath\mybibfileFive}
  }
  {
  }
  \ifdef{\mybibfileSix}
  {
    \addbibresource{\myreferencespath\mybibfileSix}
  }
  {
  }

  \ifdef{\mybibfileSeven}
  {
    \addbibresource{\myreferencespath\mybibfileSeven}
  }
  {
  }
  \ifdef{\mybibfileEight}
  {
    \addbibresource{\myreferencespath\mybibfileEight}
  }
  {
  }

  \ifdef{\mybibfileNine}
  {
    \addbibresource{\myreferencespath\mybibfileNine}
  }
  {
  }

  \ifdef{\mybibfileTen}
  {
    \addbibresource{\myreferencespath\mybibfileTen}
  }
  {
  }

  \ifdef{\mybibfileEleven}
  {
    \addbibresource{\myreferencespath\mybibfileEleven}
  }
  {
  }

  \ifdef{\mybibfileTwelve}
  {
    \addbibresource{\myreferencespath\mybibfileTwelve}
  }
  {
  }

  \ifdef{\mybibfileThirteen}
  {
    \addbibresource{\myreferencespath\mybibfileThirteen}
  }
  {
  }

  \ifdef{\mybibfileFourteen}
  {
    \addbibresource{\myreferencespath\mybibfileFourteen}
  }
  {
  }

  \ifdef{\mybibfileFifteen}
  {
    \addbibresource{\myreferencespath\mybibfileFifteen}
  }
  {
  }

  \ifdef{\mybibfileSixteen}
  {
    \addbibresource{\myreferencespath\mybibfileSixteen}
  }
  {
  }

  \ifdef{\mybibfileSeventeen}
  {
    \addbibresource{\myreferencespath\mybibfileSeventeen}
  }
  {
  }

  \ifdef{\mybibfileEighteen}
  {
    \addbibresource{\myreferencespath\mybibfileEighteen}
  }
  {
  }

  \ifdef{\mybibfileNineteen}
  {
    \addbibresource{\myreferencespath\mybibfileNineteen}
  }
  {
  }

  \ifdef{\mybibfileTwenty}
  {
    \addbibresource{\myreferencespath\mybibfileTwenty}
  }
  {
  }

  \ifdef{\mybibfileTwentyone}
  {
    \addbibresource{\myreferencespath\mybibfileTwentyone}
  }
  {
  }

  \ifdef{\mybibfileTwentytwo}
  {
    \addbibresource{\myreferencespath\mybibfileTwentytwo}
  }
  {
  }

  \ifdef{\mybibfileTwentythree}
  {
    \addbibresource{\myreferencespath\mybibfileTwentythree}
  }
  {
  }

  \ifdef{\mybibfileTwentyfour}
  {
    \addbibresource{\myreferencespath\mybibfileTwentyfour}
  }
  {
  }

  \ifdef{\mybibfileTwentyfive}
  {
    \addbibresource{\myreferencespath\mybibfileTwentyfive}
  }
  {
  }

}
{
  % Use bibtex
  \usepackage[noadjust]{cite}      % Written by Donald Arseneau
                                   % V1.6 and later of IEEEtran pre-defines the format
                                   % of the cite.sty package \cite{} output to follow
                                   % that of IEEE. Loading the cite package will
                                   % result in citation numbers being automatically
                                   % sorted and properly "ranged". i.e.,
                                   % [1], [9], [2], [7], [5], [6]
                                   % (without using cite.sty)
                                   % will become:
                                   % [1], [2], [5]--[7], [9] (using cite.sty)
                                   % cite.sty's \cite will automatically add leading
                                   % space, if needed. Use cite.sty's noadjust option
                                   % (cite.sty V3.8 and later) if you want to turn this
                                   % off. cite.sty is already installed on most LaTeX
                                   % systems. The latest version can be obtained at:
                                   % http://www.ctan.org/tex-archive/macros/latex/contrib/supported/cite/
}
