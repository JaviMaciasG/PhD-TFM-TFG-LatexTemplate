%%%%%%%%%%%%%%%%%%%%%%%%%%%%%%%%%%%%%%%%%%%%%%%%%%%%%%%%%%%%%%%%%%%%%%%%%%%
%
% Generic template for TFC/TFM/TFG/Tesis
%
% $Id: glossaries.tex,v 1.6 2015/06/05 00:10:32 macias Exp $
%
% By:
%  + Javier Macías-Guarasa. 
%    Departamento de Electrónica
%    Universidad de Alcalá
%  + Roberto Barra-Chicote. 
%    Departamento de Ingeniería Electrónica
%    Universidad Politécnica de Madrid   
% 
% Based on original sources by Roberto Barra, Manuel Ocaña, Jesús Nuevo,
% Pedro Revenga, Fernando Herránz and Noelia Hernández. Thanks a lot to
% all of them, and to the many anonymous contributors found (thanks to
% google) that provided help in setting all this up.
%
% See also the additionalContributors.txt file to check the name of
% additional contributors to this work.
%
% If you think you can add pieces of relevant/useful examples,
% improvements, please contact us at (macias@depeca.uah.es)
%
% You can freely use this template and please contribute with
% comments or suggestions!!!
%
%%%%%%%%%%%%%%%%%%%%%%%%%%%%%%%%%%%%%%%%%%%%%%%%%%%%%%%%%%%%%%%%%%%%%%%%%%%


% Define a new glossary type for symbols used in equations (example)
\newglossary[slg]{symbols}{sym}{sbl}{List of Symbols}

% Compile the glossary with an external program makeglossaries or with latex
% itself. Note that if you are on overleaf the makeglossaries program doesn't
% work unless the book.tex is in the root folder.
\ifthenelse{\equal{\glossariessystem}{makeglossaries}}
{
    \makeglossaries               % DO NOT TOUCH THIS!
}
{
    \makenoidxglossaries
}


%%% Local Variables:
%%% TeX-master: "../book"
%%% End:


