\begin{landscape}
\thispagestyle{plain}
\noindent \textbf{RÚBRICA PARA LA EVALUACIÓN DEL DESARROLLO DEL TFG}\footnote{Aprobado en Comisión de Calidad el 25 de junio de 2024}
\renewcommand{\arraystretch}{1.3} % Increases row height

\begin{table}[ht]
  \centering
  \small
\begin{tabularx}{\linewidth}{|p{7cm}|p{9cm}|p{1cm}|p{1cm}|p{5cm}|}
\cline{1-5}
\textbf{Resultados de aprendizaje} & \textbf{Criterios de Evaluación} & \textbf{Peso} & \textbf{Calif} & \textbf{Informe del Tribunal} \\ \cline{1-5}
  \multirow{7}{=}{%
  \begin{minipage}[t]{7cm}
    \textbf{BLOQUE 1: Proyecto}
    \begin{itemize}[left=5pt, itemsep=1pt, topsep=1pt]
      \item Proyecto relacionado con el ámbito con estándar de calidad adecuado (RATFG2)
      \item Aspectos regulatorios de los proyectos en el ámbito (RATFG4)
      \item Integración de competencias adquiridas (RATFG5)
    \end{itemize}
  \end{minipage}}
    & Integra competencias (CE1) &  &  & \multirow[c]{7}{=}{\centering \footnotesize Ver comentarios del informe.} \\ \cline{2-4}
    & Iniciativa, toma de decisiones y creatividad (CE2) &  &  &  \\ \cline{2-4}
    & Maneja especificaciones, reglamentos y normativa (CE3) &  &  &  \\ \cline{2-4}
    & Proyectos con estándar de calidad adecuado (CE4) &  &  &  \\ \cline{2-4}
    & Defiende un proyecto en el ámbito de la profesión (CE5) &  &  &  \\ \cline{2-4}
    & \cellcolor[HTML]{C0C0C0}Peso Máximo / Calificación & \multirow[c]{2}{=}{\centering X} & \multirow[c]{2}{=}{\centering X} &  \\
    & \cellcolor[HTML]{C0C0C0}(Peso máximo: 2 puntos) &  &  &  \\ \cline{1-5}
  \multirow{6}{=}{%
  \begin{minipage}[t]{7cm}
    \vspace{1mm}
    \textbf{BLOQUE 2: Desarrollo del trabajo}
    \begin{itemize}[left=5pt, itemsep=1pt, topsep=1pt]
      \item Interpretar, comprender y diseñar una aproximación al problema con creatividad e iniciativa (RATFG1)
      \item Trabajar de forma autónoma (RATFG10)
      \item Buscar y gestionar la información necesaria para dar respuesta a los retos del proyecto (RATFG6)
      \item Planificación de tareas (RATFG7)
    \end{itemize}
  \end{minipage}}

    & \makecell[l]{Búsqueda de información (CE6)} &  &  &  \multirow[c]{6}{=}{\centering \footnotesize Ver comentarios del informe.} \\ \cline{2-4}
    & Sintetizar la información, visión global y estado del arte (CE7) &  &  &  \\ \cline{2-4}
    & Iniciativa y capacidad para tomar decisiones y aprender de forma autónoma (CE8) &  &  &  \\ \cline{2-4}
    & Planificación del trabajo y análisis/justificación de desviaciones respecto a esa planificación (CE9) &  &  &  \\ \cline{2-4}
    & \cellcolor[HTML]{C0C0C0}Peso Máximo / Calificación & \multirow[c]{2}{=}{\centering X} & \multirow[c]{2}{=}{\centering X} &  \\
    & \cellcolor[HTML]{C0C0C0}(Peso máximo: 3 puntos) &  &  &  \\ \cline{1-5}

  \multirow{4}{=}{
  \begin{minipage}[t]{7cm}
    \vspace{-5mm}
    \textbf{BLOQUE 3: Memoria}
    \begin{itemize}[left=5pt, itemsep=1pt, topsep=1pt]
      \item Calidad del informe (memoria): estado del arte, resultados, valoración de los mismos y propuesta de mejoras (RATFG8)
    \end{itemize}
  \end{minipage}}
    & Memoria sigue la normativa (CE14) &  &  &  \multirow[c]{4}{=}{\centering \footnotesize Ver comentarios del informe.} \\ \cline{2-4}
    & Memoria y trabajo desarrollado coinciden (CE16) &  &  &  \\ \cline{2-4}
    & \cellcolor[HTML]{C0C0C0}Peso Máximo / Calificación & \multirow[c]{2}{=}{\centering X} & \multirow[c]{2}{=}{\centering X} &  \\
    & \cellcolor[HTML]{C0C0C0}(Peso máximo: 2 puntos) &  &  &  \\ \cline{1-5}
\end{tabularx}
%\caption{Rúbrica para la evaluación del desarrollo del TFG}
\label{tab:evaluacion_tfg}

\end{table}

\end{landscape}
